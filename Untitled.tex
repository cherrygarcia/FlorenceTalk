\documentclass{beamer}
\usepackage{geometry}       
\usepackage{beamer}     
%\usetheme{Warsaw}
           % See geometry.pdf to learn the layout options. There are lots.
%\geometry{letterpaper}                   % ... or a4paper or a5paper or ... 
%\geometry{landscape}                % Activate for for rotated page geometry
\usepackage[parfill]{parskip}    % Activate to begin paragraphs with an empty line rather than an indent
\usepackage{graphicx}
\usepackage{amssymb}
\usepackage{epstopdf}

\DeclareGraphicsRule{.tif}{png}{.png}{`convert #1 `dirname #1`/`basename #1 .tif`.png}

\title{Bayesian approach for addressing covariate measurement error in propensity score methods}
\author{Elizabeth Stuart and [insert other folks here]}
%\date{}                                           % Activate to display a given date or no date

\begin{document}

\begin{frame}
\frametitle{Agenda}
\begin{enumerate}
\item Background
\begin{itemize}
\item Motivation
\item Previous Research
\item Goal
\end{itemize}
\item Methods
\begin{itemize}
\item Notation
	\item Estimands and Estimators
	\item Simulation Set-Up
\end{itemize}
\item Results
\begin{itemize}
	\item Simulation
	\item Illustrative Example
\end{itemize}
\item Conclusions
\end{enumerate}
\end{frame}

  \begin{frame}
\frametitle{Motivation}

Balancing score property of propensity scores (PS) assumes that:
\begin{enumerate}
\item all confounders are observed and 
\item measured without error.
\end{enumerate}
\end{frame}

\begin{frame}
\frametitle{ Motivation}

\begin{itemize}
\item In reality, covariate measurement error may be the rule rather than the exception. 
\begin{itemize}
	\item self-reported measures: household income, weight, age of parents.
	\item imperfect instruments: blood pressure, cortisol levels.
	\item latent constructs: depression, disability.
\end{itemize}
\end{itemize}

 
\end{frame}

\begin{frame}

\frametitle{ Motivation}

\begin{itemize}
\item In reality, covariate measurement error may be the rule rather than the exception. 
\begin{itemize}
	\item self-reported measures: household income, weight, age of parents.
	\item imperfect instruments: blood pressure, cortisol levels.
	\item latent constructs: depression, disability.
\end{itemize}
\item Covariate measurement error may compromise the bias-reduction potential of propensity scores if treatment assignment depends on the true, unobserved covariate.
\end{itemize}

\end{frame} 

\begin{frame}

\frametitle{ Motivation}


\begin{itemize}
\item In reality, covariate measurement error may be the rule rather than the exception. 
\begin{itemize}
	\item self-reported measures: household income, weight, age of parents.
	\item imperfect instruments: blood pressure, cortisol levels.
	\item latent constructs: depression, disability.
\end{itemize}
\item Covariate measurement error may compromise the bias-reduction potential of propensity scores if treatment assignment depends on the true, unobserved covariate.
\item Researchers left with the choice: exclude mismeasured covariates from PS model or ignore the measurement error. 

\end{itemize}

\end{frame}

\begin{frame}
\frametitle{ Previous Research}


Focus has been on classical measurement error


$W = X + U$,  $E(U \vert X)=0$,   with constant variance $U \vert X \sim Normal(0,\sigma^2_u)$  
 

where $X$ is the correctly measured covariate, and  $W$ is the mismeasured version of $X$.

\end{frame} 

\begin{frame}

\frametitle{ Previous Research}

\begin{itemize}
\item Steiner, Cook, Shadish. 2011: Classical measurement error in covariate(s) compromises bias-reduction potential of propensity score methods.
\end{itemize}
\end{frame} 

\begin{frame}

\frametitle{ Previous Research}

\begin{itemize}
\item Steiner, Cook, Shadish. 2011: Classical measurement error in covariate(s) compromises bias-reduction potential of propensity score methods.
\item Millimet. 2010: Classical and nonclassical measurement error in covariate(s) compromises bias-reduction potential of propensity score methods. 
\end{itemize}
\end{frame} 

\begin{frame}

\frametitle{ Previous Research}

\begin{itemize}
\item Steiner, Cook, Shadish. 2011: Classical measurement error in covariate(s) compromises bias-reduction potential of propensity score methods.
\item Millimet. 2010: Classical and nonclassical measurement error in covariate(s) compromises bias-reduction potential of propensity score methods. 
\item McCaffrey, Lockwood, Setodji. 2011: Propose IPW that corrects for classical measurement error in the covariates.
\end{itemize}
\end{frame} 

\begin{frame}

\frametitle{ Previous Research}

\begin{itemize}
\item Steiner, Cook, Shadish. 2011: Classical measurement error in covariate(s) compromises bias-reduction potential of propensity score methods.
\item Millimet. 2010: Classical and nonclassical measurement error in covariate(s) compromises bias-reduction potential of propensity score methods. 
\item McCaffrey, Lockwood, Setodji. 2011: Propose IPW that corrects for classical measurement error in the covariates.
\item Lockwood, McCaffrey. 2014: Argue that PS matching using covariates measured with error (only) will not work, but suggest that using the covariates measured with error in conjunction with treatment status may work in some scenearios.
\end{itemize}
\end{frame} 

\begin{frame}

\frametitle{ Previous Research}

\begin{itemize}
\item Steiner, Cook, Shadish. 2011: Classical measurement error in covariate(s) compromises bias-reduction potential of propensity score methods.
\item Millimet. 2010: Classical and nonclassical measurement error in covariate(s) compromises bias-reduction potential of propensity score methods. 
\item McCaffrey, Lockwood, Setodji. 2011: Propose IPW that corrects for classical measurement error in the covariates.
\item Lockwood, McCaffrey. 2014: Argue that PS matching using covariates measured with error (only) will not work, but suggest that using the covariates measured with error in conjunction with treatment status may work in some scenearios.
\item Raykov. 2012: Propose latent variable approach to address covariate measurement error in propensity score methods. Assumes have congeneric measures for each covariate measured with error. 
\end{itemize}

\end{frame} 

\begin{frame}

\frametitle{ Research Gap}


Non-classical measurement error: differential by treatment status.

\begin{itemize}
\item Systematic differential measurement error that affects the mean. 
\begin{itemize}
	\item Example: Adolescents in disadvantaged neighborhoods (`treatment' group) tend to overestimate their mothers' age when the adolescents were born.
\end{itemize}
\item Heteroscedastic differential measurement error that affects the variance.
\begin{itemize}
	\item Example: Adolescents in disadvantaged neighborhoods are less accurate in knowing their mothers' age when the adolescents were born.
\end{itemize}
\end{itemize}
 

\end{frame} 

\begin{frame}

\frametitle{ Goal}


Approach that can flexibly handle covariate measurement error that is differential by treatment status.



Bayesian approach
\begin{itemize}
	\item Most flexible approach for addressing measurement error (Carroll et al., 2006). Especially useful for measurement error model involving heteroscedasticity.
	\item Propogates uncertainty.
	\item Appropriate when validation data are external to the study sample instead of internal (Cole et al., 2006).
	\item Maximum likelihood approach has similar advantages, but Bayesian is simpler to implement (Hossain, Gustafson, 2009).
\end{itemize}

 

\end{frame} 

\begin{frame}

\frametitle{ Notation}


Let observed data $O=(W, Y, A, Z)$ and complete data $C=(W, Y, A, X, Z)$, where:


\begin{itemize}
\item $Y$ = observed, continuous outcome of interest. 

\item $A$ = observed, binary (0/1) variable indicating treatment. 

\item $Z$ = observed, continuous covariate. 

\item $X$ = unobserved, continuous covariate. 

\item $W$ = observed, mismeasured version of $X$, where the mismeasurement depends on the tratment. $W \sim Normal(f(X,A), \sigma^2f(X,A)^2)$

\end{itemize}


\end{frame} 






\end{document}
